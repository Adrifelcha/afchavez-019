\documentclass[jou,apacite]{apa6}

\usepackage{hanging}
\usepackage[utf8]{inputenc} % Required for inputting international characters


\title{Exploring the extensiveness of the Mirror Effect to the application of SDT model out of Recognition Memory}
\shorttitle{APA style}

\twoauthors{Adriana F. Ch\'{a}vez}{J.M. Ni\~{n}o}
\twoaffiliations{National Autonomous University of Mexico}{National Autonomous University of Mexico}

\abstract{Within recognition memory studies where Signal Detection Theory has been applied to describe subjects’ performance, a pattern of responses known as the Mirror Effect has shown that when comparing subjects’ performance between classes of stimuli that are differentially recognized, this difference appears both for the identification of known and new items. However, the extensiveness of this pattern to other fields has not been explored yet. By using what is known about the Ebbinghaus illusion to design two levels of discriminability, evidence of the Mirror Effect in a detection task, confidence ratings included, that involves perception only is shown.}

\rightheader{APA style}
\leftheader{Author One}

\begin{document}
\maketitle 

\section{Introduction}

Signal Detection Theory (SDT) provides a statistical model that has been broadly used to describe detection as one of the major adaptative problems that organisms have to face in order to allocate their behavior with optimability. The core idea behind the definition of detection tasks as an adaptative problem is that organisms habitate in an uncertain world, where they are constantly exposed to a wide variety of stimulation which may, or may not, give them relevant information about the structure of their environment and the operating rules. Once the organism has learned that certain stimuli are related to the occurence of (or to access to) a biologically relevant event, they become 'signals' which the organism needs to detect in order to be able to adapt their behavior to the contingencies announced by them. \\

SDT is a powerful tool for describing all kinds of situations in which an organism faces the need to detect a certain 'signal', discarding every other stimuli ('noise') present during the task. There are two main assumptions upon which SDT bases all its framework: 1) there is always some degree of uncertainty (which is represented by the) and 2) consequences matter: making a right detection judgement may lead to some reward while failing has a cost too, and more importantly, the scale in which these things happen tend to be different per outcome (For example, for an animal trying to detect the presence of predators in its environments, making a False Alarm may lead to an unnecessary energy loss, but making a Miss could mean death). Organisms try to compensate the uncertainty contained within their detection task by taking into account every piece of information they have about their environment (in terms of the consequences at risk and the probabilistic structure of the task) to fix a criterion over the Evidence axis, which is going to be used to determine..\\

Ever since SDT was proposed within the engineering field to be first applied to a psychological context by the end of that same year, SDT has been used among an incredibly wide variety of phenomena and detection tasks within and outside of Experimental Psychology. SDT has been used as a framework for both, describing detection tasks and conducting data analysis...\\

When SDT has been applied to guide the data analysis of recognition memory experiments where subjects' performance is compared across two classes of stimuli A and B (in which the latter is known to be more difficult to recognize than the first ($d'(A)>d'(B)$)),

Al aplicar el modelo de Detección de Señales a tareas de Memoria de Reconocimiento, donde los participantes tienen que identificar los elementos ya antes vistos (la Señal) dentro de un conjunto de ítems que incluye elementos presentados en una fase previa y elementos nuevos (el Ruido), se ha encontrado consistentemente un patrón de respuestas cuando se compara la ejecución de los participantes entre dos clases de estímulos, (siendo una de ellas más fácil de reconocer (A) que la otra (B)), que demuestra que los participantes no sólamente son mejores reconociendo los elementos previamente mostrados en la condición A ($Hits(A)>Hits(B)$), sino que también son mejores identificando los estímulos nuevos dentro de esta misma condición ($F.alarm(A)<F.alarm(B)$). Dado que en estos estudios los participantes experimentales no saben que se ha incluido más de una clase de estímulos en la tarea que se les presenta, se asume que utilizan un sólo criterio de elección para emitir sus respuestas y de acuerdo a las tasas reportadas de Hits y Falsas Alarmas por cada clase, se sugiere que las distribuciones de Ruido y Señal de cada clase se distribuyen a lo largo del mismo eje de evidencia de tal forma que parecieran reflejarse entre sí. Por ello, en la literatura en Memoria de Reconocimiento se ha identificado dicho patrón de respuestas bajo el nombre de Efecto Espejo.\\

%3.5*42.5
El Efecto Espejo sólo ha sido estudiado dentro del dominio de la Memoria de Reconocimiento, donde se ha reportado evidencia de su existencia a lo largo de una amplia variedad de procedimientos (tareas Sí/No, tareas de Elección Forzada entre dos Alternativas y protocolos con Escala de Confianza) y variables (palabras comunes vs palabras extrañas; estímulos abstractos vs estímulos concretos; palabras escritas al revés vs palabras bien escritas; imágenes a color o en blanco y negro, etc). Como resultado, gran parte de los modelos y teorías desarrollados para dar cuenta de este fenómeno tienden a hacerlo en términos de la estructura propia de las tareas de reconocimiento, donde se incluye una fase de estudio en la que se asume que los participantes procesan los estímulos para añadirles la \textit{'familiaridad'} necesaria para poder reconocerlos en una segunda fase y donde se asume que tienen origen las diferencias observadas en el desempeño de los participantes -en otras palabras, se asume que las clases de estímulos puestas a prueba son procesadas y atendidas de distinta manera durante la fase de estudio previa a la tarea de reconocimiento-.\\

El interés principal del presente trabajo de tesis fue explorar la generalizabilidad del Efecto Espejo, buscando evidencia del mismo en una tarea de detección ajena a la Memoria de Reconocimiento. Para ello, se presentan dos variaciones de una tarea de detección perceptual (visual) que emula la estructura de los estudios donde dicho fenómenos ha sido reportado en Memoria, construyendo dos niveles de dificultad con base en la literatura en Ilusiones Ópticas. La tarea propuesta fue presentada a los participantes a partir de dos protocolos: una tarea Sí/No y la asignación de un puntaje en una Escala de Confianza. Los resultados e implicaciones de los mismos se discuten en detalle.\\

Most of the theories and explanations developed to account for the Mirror Effect response patterns tend to do it in terms of the superior cognitive processes that are thought to be involved in recognition memory, or even furthermore, in terms of the specific structure of recognition tasks (where parti)


\section{Method}

\subsection{Experiment 1: A (perceptual) detection task}

\section{Results}

\subsection{Experiment 1}

\section{Discussion}



\section{References}

\begin{hangparas}{.3in}{1}

Gallistel, C. R., Krishan, M., Liu, Y., Miller, R., \& Latham, P. E. (2014). The perception of probability. {\it Psychological review, 121(1),} 96.

Kalman, R. E. (1960). A new approach to linear filtering and prediction problems. {\it Journal of basic Engineering,} 82(1), 35-45.

Miller R.R., Barnet R.C. \& Grahame NJ (1995) Assessment of the Rescolra-Wagner model. {\it Psychological Bulletin} 117: 363–386.

Nassar, M. R., Wilson, R. C., Heasly, B., \& Gold, J. I. (2010). An approximately Bayesian delta-rule model explains the dynamics of belief updating in a changing environment. {\it Journal of Neuroscience,} 30(37), 12366-12378.

Rescorla R.A. \& Wagner A.R. (1972) A Theory of Pavlovian conditioning: variations in the effectiveness of reinforcement and nonreinforcement. In: Black AH, Prokasy WF, editors, Classical conditioning II: current research and theory. New York: Appleton Century Crofts. chapter 3. pp. 64–99. 

Ricci, M. \& Gallistel, R. (2017). Accurate step-hold tracking of smoothly varying periodic and aperiodic probability.{\it Attention, Perception, \& Psychophysics, }1-15.

Schultz, W., Dayan, P., \& Montague, P. R. (1997). A neural substrate of prediction and reward. {\it Science,} 275(5306), 1593-1599.

Speekenbrink, M., \& Shanks, D. R. (2010). Learning in a changing environment. {\it Journal of Experimental Psychology: General,} 139(2), 266.

Sutton, R. S. (1992). Gain adaptation beats least squares. In {\it Proceedings of the 7th Yale workshop on adaptive and learning systems} (Vol. 161168).

Sutton, R. S., \& Barto, A. G. (1998). {\it Introduction to reinforcement learning (Vol. 135)} . Cambridge: MIT Press.

Wilson, R. C., Nassar, M. R., \& Gold, J. I. (2013). A mixture of delta-rules approximation to Bayesian inference in change-point problems. {\it PLoS computational biology,} 9(7), e1003150.




\end{hangparas}


\onecolumn

\bibliography{sample}

\end{document}