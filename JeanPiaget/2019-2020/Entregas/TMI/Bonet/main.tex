\documentclass{article}
\usepackage[utf8]{inputenc}

\title{¿Quien o que es Dow Jones?}
\author{Emiliano Bonnet }
\date{October 2019}

\begin{document}

\maketitle

\section{Introducción}
Dow Jones es una compañía financiera la cual creó el índice bursátil “Dow Jones” , y esta misma empresa es responsable de cientos de índices más. Es fácilmente aplicado en la bolsa tanto mexicana como estadounidense, este fue creado por dos periodistas americanos. Este refleja el comportamiento del precio de la acción de más de 30 compañías industriales más importantes de todo Estados Unidos por lo cual nos afecta directamente. Este tema se eligió debido a la falta de cultura financiera en Mexico, y de quiere dar a entender el funcionamiento de las acciones y de la bolsa como tal, y como las acciones de una compañía extranjera puede afectarnos. El índice Dow Jones se divide en 3 sub indices los cuales corresponden a Índices de Utilidades, transporte y compuestos 
PREGUNTA DE INVESTIGACION: ¿Que es y como funciona la bolsa estadounidense?

OBJETIVOS:

Objetivo general: Dar a conocer la importancia de la bolsa de Estados Unidos y su funcionamiento.

Objetivo Específico: Evidenciar la importancia de las industrias Dow Jones y su relación con la bolsa.

JUSTIFICACIÓN: Este tema se seleccionó cuidadosamente para poder exponer el funcionamiento de la bolsa estadounidense de una manera sencilla. La bolsa es un tabú en Mexico debido a que la gente no invierte en acciones, esto se debe a una falta de educación financiera, ya que la cultura popular ha dado a entender que para poder invertir en la bolsa se debe ser un alto financiero o incluso un genio, en este trabajo se dará a conocer el funcionamiento de la bolsa, las acciones, las inversiones y como la bolsa de Estados Unidos nos puede llegar a afectar. Se expondrá como todo lo anteriormente mencionado tiene una relación con el índice Bursátil Dow Jones, con en fin de poder explicar las inversiones y el por qué es importante hacerlas 



\end{document}
